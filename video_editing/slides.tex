\documentclass[aspectratio=169, 20pt]{beamer}

\title{Language Bootstrapping: Learning Word Meanings From Perception-Action Association}
\author{Giampiero Salvi, Luis Montesano, Alexandre Bernardino, José Santos-Victor}
\date{}

\usepackage{tikz}
\usetikzlibrary{shapes.geometric}
\usetikzlibrary{fit}
\tikzstyle{circle}=[shape=circle,minimum size=0.7cm,very thick]
\tikzstyle{every path}=[very thick]
\usetikzlibrary{arrows}
\usetikzlibrary{positioning}


% The following command suppresses all navigation symbols:
\setbeamertemplate{navigation symbols}{}
\setbeamercolor{background canvas}{bg=black}
\setbeamercolor{normal text}{fg=white}
\setbeamercolor{frametitle}{fg=white}
\setbeamercolor{title}{fg=white}
\usebeamercolor[fg]{normal text}
\setbeamertemplate{frametitle}[default][center]

\begin{document}
\begin{frame}
  \maketitle
\end{frame}

\begin{frame}{LEARNING AFFORDANCES}
\large After gaining some knowledge about individual object and effect properties, the robot learns to associate the features of objects to the consequences (effects) of its actions, thus creating a model of object affordances
\end{frame}

\begin{frame}{\large LEARNED AFFORDANCE NETWORK}
  \begin{center}
    \resizebox{!}{0.65\textheight}{
    \tikzstyle{affnode} = [ellipse, draw=gray!60, text=black, fill=gray!60, ultra thick, font=\bf]%fill=gray!20, thick]%node distance=1cm, text width=6em, text centered, minimum height=4em, thick]
    \tikzstyle{affarrow} = [->, ultra thick, >=stealth']%, shorten >=2pt]%, shorten >=2pt]
    \centering
    \begin{tikzpicture}[node distance=1cm and 2cm]
      % single nodes
      \node[affnode] (action) {Action};
      \node[affnode, right= of action] (size) {Size};
      \node[affnode, right= of size] (shape) {Shape};
      \node[affnode, right= of shape] (color) {Color};
      \node[affnode, below= of action] (handvel) [xshift=3cm] {Hand Vel};
      \node[affnode, right= of handvel] (objvel) {Obj Vel};
      \node[affnode, below= of handvel] (objhandvel) [xshift=3cm] {Obj Hand Vel};
      \node[affnode, below= of objhandvel] (contact) {Contact};
      % arrows
      \draw[affarrow] (action) -- (handvel);
      \draw[affarrow] (action) -- (objvel);
      \draw[affarrow] (action) to [out=260,in=170] (objhandvel);
      \draw[affarrow] (action) to [out=230,in=160] (contact);
      \draw[affarrow] (size) -- (objvel);
      \draw[affarrow] (shape) -- (objvel);
      \draw[affarrow] (shape) to [out=330,in=20] (contact);
      \draw[affarrow] (objvel) -- (objhandvel);
    \end{tikzpicture}
    }
  \end{center}
\end{frame}

\begin{frame}{LEARNING WORD-TO-MEANING ASSOCIATIONS}
\large While listening to context related verbal descriptions of the task provided by humans, the robot can learn to associate words to perceptions and actions by measuring their co-occurence.
\end{frame}

\begin{frame}
  \large After a training stage, the robot has learned the following word-to-meaning associations.

  \vspace{1cm}
  White nodes: words
  
  Gray nodes: meanings
\end{frame}
\end{document}
